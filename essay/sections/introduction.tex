% !TeX root = ..\essay.tex

\section{Introduction}
Automatic text summarization is the task of automatically creating summaries from
given text sources. For the project described in this report the process was
divided into three pipeline steps:
\begin{itemize}
    \item Step 1 - Content Selection:  \\ The task here is to select relevant
        nuggets (sentences or part of sentences) out of the input. The
        input consisted of important sentences that were extracted from
        documents and covered 49 topics.
    \item Step 2 - Creating Hierarchies: \\ The task of this step is to create
        hierarchies that represented the structure of the nuggets from the
        previous step and ordered them into subtopics.
    \item Step 3 - Creating Summaries from Hierarchies: \\ In the last step the
        hierarchies of the second step have to be converted into short summaries.
        The summaries could be either about the complete topic or subtopics.
\end{itemize}
Our group chose to deal with all three pipeline steps and create general summaries
without focusing on just a few subtopics. The implementation process and
the results will be further explained in the following chapter. In
chapter~\ref{ch:evaluation} information about the quality of the resulting
summaries are also given in the form of an evaluation.