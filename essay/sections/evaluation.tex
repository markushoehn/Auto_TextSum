% !TeX root = ..\essay.tex

\section{Evaluation}
\label{ch:evaluation}

\subsection{evaluation setting}
To evaluate the summaries they were manually annotated.
Each summary has been annotated by at least four raters with the following criteria:
\begin{table}[H]
	\begin{tabularx}{\textwidth}{l|X} \toprule
		criteria & description \\ \midrule
		Grammaticality      & \\
		Non-Redundancy      & \\
		Referential Clarity & \\
		Focus               & \\
		Structure           & \\
		Coherence           & \\
		Readability         & \\
		Information Content & \\
		Spelling            & \\
		Length              & \\
		Overall Quality     & \\ \bottomrule
	\end{tabularx}
	\caption{evaluation criteria}
	\label{tab:evacriteria}
\end{table}

The score for each criteria was set with the help of a five-point Likert scale.
For each criteria the opportunity to give an estimation in weighting and confidence was be expected, too. These further possibilities to rate a summary are also realized with a five-point Likert scale.
In table~\ref{tab:evalikert} the scales for Score, Weight and Confidence are shown:

\begin{table}[H]
	\begin{tabularx}{\textwidth}{l|XXX} \toprule
		Scale & Score & Weight & Confidence \\ \midrule
		1 & very poor & completely unimportant & very low \\
		2 & poor & unimported & low \\
		3 & barerly acceptable & indifferent & half sure \\
		4 & good & important & high \\
		5 & very good & absplutely important & very high \\ \bottomrule    
	\end{tabularx}
	\caption{Likert scale for Weight and Confidence}
	\label{tab:evalikert}
\end{table}

The annotators had also the opportunity to comment each criteria for each summary with free text.

\subsection{JSD}

\begin{figure}[H]
	\centering
	\includegraphics[trim=0 150 0 150, width=\textwidth]{img/jsd.pdf}
	\caption{JSD score}
	\label{fig:jsd}
\end{figure}


\subsection{Scores per Criteria}

First we analized our average score per criteria against the average score per critera over all groups. The results are shown in figure~\ref{fig:spc}.

\begin{figure}[H]
	\centering
	\includegraphics[trim=0 150 0 150, width=\textwidth]{img/scores_per_criteria.pdf}
	\caption{scores per criteria}
	\label{fig:spc}
\end{figure}

There are only two criteria our group is signifiant worse than the average, "non-redundancy" and "length". On the other hand there are several criteria we are better than the average, e.g. "referential clarity" or "focus".

For a better understanding of these values, we have to make a closer look at the free text comments.
In table~\ref{tab:evacomments} you can see an overview with all criteria, the number of comments and the main points.

\begin{table}[H]
	\begin{tabularx}{\textwidth}{llX} \toprule
		criteria & \# & main point \\ \midrule
		Grammaticality      & 25 & puncuation incl. periods, parenthesis  \\
		Non-Redundancy      & 19 & repetition resp. multiple definitions \\
		Referential Clarity & 11 & unsolved reference \\
		Focus               & 17 & one sentence does not fit \\
		Structure           & 13 & order of sentences \\
		Coherence           & 9  & lack of information due to bad sentence connection \\
		Readability         & 21 & punctuation and too long sentences \\
		Information Content & 15 & one sentence does not fit \\
		Spelling            & 17 & punctuation and case-related problems \\
		Length              & 31 & not exactly 600 characters \\
		Overall Quality     & 14 & lack of information, structure, or length \\ \bottomrule
	\end{tabularx}
	\caption{evaluation criteria}
	\label{tab:evacomments}
\end{table}

criteria not independent.
punctuation often written for multiple criteria, readability as result of structure, coherence, spelling, grammar.

overall should be average of all but only in n cases its really true.

You can see that the annotaters give more comments for criteria we are not so good at as in other ones.


\subsection{box plot}

\begin{figure}[H]
	\centering
	\includegraphics[trim=0 150 0 150, width=\textwidth]{img/box.pdf}
	\caption{statistical values over all topics per group}
	\label{fig:svg}
\end{figure}


\subsection{Score Calculation with all scales}

\begin{figure}[H]
	\centering
	\includegraphics[trim=0 150 0 150, width=\textwidth]{img/score_per_topic.pdf}
	\caption{scores per topic}
	\label{fig:spt}
\end{figure}


\subsection{calculated score per topic - sorted}

\begin{figure}[H]
	\centering
	\includegraphics[trim=0 150 0 150, width=\textwidth]{img/score_per_topic_sorted.pdf}
	\caption{scores per topic sorted ascending}
	\label{fig:spts}
\end{figure}

\subsection{best and worst}