% !TeX root = ..\essay.tex

\section{Evaluation}
\label{ch:evaluation}

\subsection{evaluation setting}
To evaluate the summaries they were manually annotated.
Each summary has been annotated by at least four raters with the following criteria:
\begin{itemize}
	\item Grammaticality \\ jedes item erklären
	\item Non-Redundancy
	\item Referential Clarity
	\item Focus
	\item Structure
	\item Coherence
	\item Readability
	\item Information Content
	\item Spelling
	\item Length
	\item Overall Quality
\end{itemize}

The score for each criteria was set with the help of a five-point Likert scale with the following scales:

\begin{enumerate}
	\item very poor
	\item poor
	\item barely acceptable
	\item good
	\item very good
\end{enumerate}

For each criteria the opportunity to give an estimation in weighting and confidence would be expected, too.
These further possibilities to rate a summary are also realized with a five-point Likert scale.
In table~\ref{tab:evalikert} the scales for Weight and Confidence are shown:

\begin{table}[H]
	\begin{tabular}{l|ll} \toprule
		Scale & Weight & Confidence \\ \midrule
		1 & completely unimportant & very low \\
		2 & unimported & low \\
		3 & indifferent & half sure \\
		4 & important & high \\
		5 & absplutely important & very high \\ \bottomrule    
	\end{tabular}
	\caption{Likert scale for Weight and Confidence}
	\label{tab:evalikert}
\end{table}

The annotators had also the opportunity to comment each criteria for each summary with free text.

\subsection{JSD}
\subsection{Scores per Criteria}

\subsection{Score Calculation with all scales}
\subsection{box plot}
\subsection{calculated score per topic - sorted}

\subsection{best and worst}