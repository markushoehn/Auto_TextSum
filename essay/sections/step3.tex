% !TeX root = ..\essay.tex

\subsection{Step 3 - Creating Summaries from Hierarchies}
After the second step of the pipeline our system will have constructed hierarchies,
as explained in the previous chapter. To summarize the result of the step: Each
hierarchy file contains the relevant sentences of one of the topics. The hierarchies
themselves consist of XML trees which should represents subtopics of the topics.
The bubbles, which are nodes of the trees, can have multiple nuggets, which in
our system are always complete sentences. The task of the third pipeline step
was now to construct summaries out of these hierarchy files. We focused on
creating general summaries over the complete topic instead of only focusing on
subtopics. The challenge is here the length of the summaries, because they should
contain only about 600 characters. It is therefore important to select
representative sentences and arrange them in a meaningful way, so that the end
result has a high quality. The quality of each summary was later rated by members
of other teams regarding the following criteria:
\begin{itemize}
    \item Grammaticality
    \item Non-Redundancy
    \item Referential Clarity
    \item Focus
    \item Structure
    \item Coherence
    \item Readability
    \item Information Content
    \item Spelling
    \item Length
    \item Overall Quality
\end{itemize}
The result of this rating is given in chapter~\ref{ch:evaluation}.

Because our system uses complete sentences as nuggets, grammatically and spelling
where not a big problem, at least when the source texts were written well. The
first thing we concentrated on was to select sentences in a way that the structure
of the summary feels natural and focused. To achieve this we arrived at the
concept of traversing a hierarchy tree and it's nuggets in the same way a depth
first search would do it. The idea behind this approach is that it resembles the
way a human would talk about topics and subtopics. It starts with a really general
explanation of a topic and becomes more and more specific, before than switching
to other subtopics, becoming there more and more specific and doing the same for
other topics. When the end result of the second pipeline step has a high quality
and therefore a good structure of the sentences, our summaries would automatically
also be really good. The challenge was now of course to create summaries that
have only a specific length. We cut the deeper bubbles of the hierarchy until the
nuggets would sum up to the requested summary length. This means that the summaries
would still contain the general information about the subtopics but not became
as specific about each.

After it was revealed that the amount of characters per summary should not be
higher that 600, our approach had to be changed because such very short summaries
would result in the depth first search only looking into one subtopic, and even
there just in the general nuggets, without reaching the other subtopics. Our new
approach was than to order the trees according to their size and selecting from
the largest trees nuggets of the roots. The idea behind this is that a large tree
size (measured by the amount of nuggets) means that the subtopic represented by
this tree contains a lot of information and should therefore be included in the
summary. From each tree root we selected the shortest sentence and included it
into the summary. By doing this we could cover more subtree and had therefore a
wider range of information that could be covered. Selecting the sentences from
the roots was done because for such a short summary it would not make sense to
go into detail and take sentences from deeper within the hierarchies. This new
approach still resulted in good summaries. They where not so much focused and so
good structured anymore as before, because they jumped from subtopic to subtopic
but for a short summaries about the complete topic this is of course unavoidable.